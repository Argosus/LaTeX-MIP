\documentclass[a4, conference]{IEEEtran}
\usepackage{graphicx}
\usepackage{cite}
\usepackage{url}
\usepackage[cp1250]{inputenc}

\begin{document}

\title{From ambiguous words to key-concept extraction}

\author{
\IEEEauthorblockN{M�rius �ajgal�k, Michal Barla, M�ria Bielikov�}
\IEEEauthorblockA{Faculty of Informatics and Information Technologies\\
Slovak University of Technology\\
Ilkovi�ova, 842 16 Bratislava, Slovakia\\
Email: {fsajgalik, barla, bielik} @fiit.stuba.sk}
}


\maketitle


\begin{abstract}
Automatic acquisition of keywords for given document
is still an area of active research. In this paper, we consider
shift from keyword-based representation to other perspective
on representation of document�s focus in form of key-concepts.
The advantage of using concepts over simple words is that
concepts, apart from words, are unambiguous. This leads to better
understanding of key-concepts than keywords. We present novel
method of key-concept extraction, which provides an efficient way
of automatic acquisition of key-concepts in machine processing.
We evaluate our approach on classification problem, where we
compare it to baseline TF-IDF keyword model and present its
competitive results. We discuss its potential of its utilisation on
the Web.
\end{abstract}

\IEEEpeerreviewmaketitle

\section{INTRODUCTION}
In this modern information age, we are overflowed with
huge amounts of data. The Indexed Web is estimated to contain
over 14:35 billion web pages1 and is still growing. We strive
to develop efficient ways of organising the data. It is also
one of the goals of Semantic Web to provide such semantic
metadata for web pages, which would help the machines
to understand the semantics of the web page focus. There
have been multiple initiatives like Microformats, Microdata,
or RDF. However, there is still not enough explicit semantic
information of sufficient quality included in the web page
content2, which often forces us to incorporate some kind of
ontology to understand the content of the �wild web� better
[6].
\\

Most commonly used are still the �old classic� keywords
and description metatags, the former representing traditional
keywords most relevant for given web page and the later some
short natural language text describing the web page content.
These metatags provide a way by which computers can categorise
the content of web pages. Keyword representation of
documents is rather old and still widely used [8], [13], [22].
\\

In our work, we take another approach to automatically
extract some metadata from raw text. We do not focus on
extracting keywords, we extract key-concepts instead. The use
of key-concepts as the most relevant concepts is advantageous
in that meaning of each concept is very well-defined in an
exact and clear way. This implies another advantage over
keywords, which is greater information content of concept
vector than keyword vector. This claim is based on work
of [18], which shows that concept vector representation has
greater information content than simple words or TF-IDF
vector. To delve a little deeper, the computation of information
content of vector representation is based on mutual information
between vector items (in our case concepts) and documents,
which has been shown to have positive impact on performance
in multiple information retrieval tasks [14].
\\

To obtain concepts from plain text, we utilise WordNet
[16], which can be considered as a form of a lightweight
ontology. It should be clarified that in fact, not all WordNet
synsets are concepts. Some of them represent instances of
concepts. Author of [4] proposes a method how to distinguish
concepts from their instances in WordNet hierarchy. Although
outdated in present, since starting from version 2:1 WordNet
differentiate hypernym (hyponym) and instance hypernym (instance
hyponym) relations [17], it points out rather intuitive
observation that vertices corresponding to concept instances
always lie on the bottom of the hierarchy. In our approach,
we do not consider instance hypernym relations, thus cutting
off concept instances, which results in acquisition of pure
concepts.
\\

In this paper, we present the notion of key-concepts and
propose novel method of discovering them in text. In evaluation,
we focus on classification problem. As we already mentioned,
an efficient classification is very important in today�s
world of big data. Via classification, we demonstrate the main
power of our extracted key-concepts, which is a substantial
dimensionality reduction of document�s feature space, while
achieving greater information content than simple keywords.

\section{RELATED WORKS}

To be able to extract some concepts, we need to disambiguate
the meaning of words. There are multiple methods
for word sense disambiguation [1], [11], [12]. Some of them
also use WordNet [2], [5] to infer the most probable word
sense. Approach in [18] is little different from others, since it
does not rely on making hard decisions, but ranks the WordNet
synsets by relevance to the text (soft sense disambiguation). Its
authors describe the representation of text by WordNet synsets
instead of words. They point out two major drawbacks of �bag
of words� representation polysemy and synonymy. Polysemy
cause the ambiguity of the words since a single word can
have multiple meanings. In case of synonymy, several words
can have the same meaning and bag of words just lack the
information about such relations among them. On the other
hand, in synset representation, each synset has unique and
clear meaning. Evaluation in [18] compares several different
approaches to rank synsets in order to infer the most probable
meaning and shows that the best results are achieved by using
the PageRank algorithm.
\\

Use of PageRank algorithm [9] in text mining has been
researched by multiple researchers [23], [10], [7]. One of the
pioneer methods is TextRank [15], which is an unsupervised
algorithm for keyword extraction. Its promising results even
caused its authors to apply for a patent on it. Another example
of utilising PageRank is in [3], where it is also used to word
sense disambiguation.

\section{KEY-CONCEPT EXTRACTION}

Our method is based on disambiguating the word senses
using PageRank algorithm. The principal idea of our approach
is to infer the relevance of latent concepts hidden in text
by observing words. To infer the relevance, we take into
account several factors like collocations of words, hypernym
and holonym relations and information content of concepts.
\\

To preprocess the text, we perform part-of-speech (POS)
tagging3 to choose only the nouns (including compound nouns)
as candidates. With these feasible terms extracted, we take
all the noun synsets of WordNet, which contain at least
one of these feasible terms. We call these synsets the basis
synsets. Then we create the concept graph G = (V;E),
where vertices V are all the basis synsets plus those reachable
by following hypernym or holonym relations. This aims to
influence also the more general concepts (WordNet synsets) to
get to the broader topics discussed in the extracted article. We
do not consider instance hypernym relations, since without it
we observed better results. This makes our extracted synsets
proper concepts, since concept instances are not propagated.
These hypernym and holonym relations constitute the graph�s
edges E.
\\

After we have built the concept graph, we perform PageRank
algorithm to infer the relevance of individual concepts.
Since the concept graph is undirected and weighted, we
adapted the combination of PageRank modifications for both
undirected and weighted graphs presented in [15] (see Equation
1).
\\

The principle of our proposed approach is to do a two-pass
ranking. In the first pass, we propagate the authority of a synset
to all hypernyms and holonyms via existing edges to obtain the
most probable word senses. In the second pass, we enrich the
concept graph with edges representing the collocation relations
too (besides the hierarchical links). That is, after performing
the page ranking in the first pass, we link also the synsets that
contain some neighbouring terms in the second pass as well
to support the collocated word senses and thus, get the key
concepts. We adapted this idea from TextRank [15], which is
an unsupervised method to extract keywords. Such propagation
of collocation relations can be seen as if terms (synsets) were
voting for their neighbours. The inference is done iteratively
using formula in Equation 1 to compute a new vertex score.
\\

ROVNICA CISLO 1
\\

There, V S(v) denotes the vertex score of vertex v, Adj(v)
denotes the set of all adjacent vertices to v. Edge between
vertices Vi and Vj is weighted with value wij and d is the
damping factor usually set to 0:85 [9], [15]. We note that this
equation variant differs from that presented in [15], where
its authors consider only binary valued edge weights. We
use a little more general formula, which allows us to tweak
these values a little, since in our case, we have multiple
possible relations between concepts (hypernyms, holonyms),
not just word collocations. We observe more favourable results
with higher weights assigned to hypernym-hyponym relations,
although we did not conduct any experiments to support
this statement. For reference, we set edge weights to 1 for
hypernym-hyponym relations and 0:7 for holonym-meronym
relations. It should be noted that SemanticRank [21] is based
exactly on manipulating these weights to extract keywords and
key-sentences from text.
\\

After each run of PageRank algorithm, we do not take
just the vertex values of graph. We multiply the vertex score
obtained from both runs of PageRank by the information
content of the corresponding concept to get the final ranking.
\\

A. \textit{Notion of information content}
\\

We consider the information content [19] of concepts to
account for different commonness of different concepts. The
information content is a measure of specificity for a concept.
The higher value of information content, the more specific is
the concept (e.g. violin), whereas the lower values signify
more general concepts (e.g. object). The information content
IC(c) of a concept c is defined as the negative logarithm of
the probability of this concept:
\\

ROVNICA CISLO 2
\\

The probability of a concept P(c) is computed as relative
frequency of it:
\\

ROVNICA CISLO 3
\\

In general, N is the total number of nouns observed in
some text corpus and freq(c) denotes the concept frequency:
\\

ROVNICA CISLO 4
\\

There, words(c) is the set of words assigned to the Word-
Net concept c and count(n) is the total number of occurrences
of the noun n. In our computations, we used Google N-gram
corpus4 to compute concept probabilities.
\\

The utilisation of information content in the key-concept
computations above turns out to be quite intuitive. The information
content as a measure of concept specificity in context of
key-concept extraction is pretty analogical to inverse document
frequency as a measure of word specificity in context of
keyword extraction. To see this analogy better, we can write
the inverse document frequency as:
\\

TABULKA 1
\\

TABULKA 2
\\

ROVNICA CISLO 5
\\

As we can see, the main distinction is that the inverse document
frequency is proportional to the probability of occurrence
of word w within a document, whereas the information content
is proportional to the overall probability of a concept c. More
detailed discussion on TF-IDF from probabilistic perspective
can be found in [20].
\\

\section{EVALUATION}

We empirically evaluated the quality of extracted keyconcepts
in classification of plain texts. We used 20 newsgroups
dataset5, which is commonly used in text classification.
It contains 20 topic categories, each with 1000 documents,
yielding 20; 000 documents altogether. As a baseline to compare
to, we used standard TF-IDF vector representation. We
evaluated the performance of two standard classifiers - k-
nearest neighbours and Na�?ve Bayes classifier. We can see
detailed results of achieved classification accuracy in Figure
1 (on vertical axis, there are classification categories).
\\

According to these results, our method achieves better classification
accuracy than the standard TF-IDF. In construction
of TF-IDF vector, we do some common preprocessing. We
remove English stopwords, do POS-tagging to get only nouns
(including compound nouns) as candidate words and we use
Porter stemmer to obtain stems6. Finally, we prune some very
common (corpus frequency above 30%) and very rare (corpus
frequency below 3%) words. We should point out that TFIDF
vector contains substantially greater number of words
compared to at most 20 key-concepts obtained by our method.
\\

We experimented with different number of extracted keyconcepts.
It turns out that we succeed to extract the most
relevant concepts as the very first few ones, which we can
see from results in Table I. Even for only top 5 extracted
key-concepts, the accuracy scarcely changes and even for just
top 3 key-concepts, the results are still reasonable compared
to baseline. We used Na�?ve Bayes classifier to obtain these
results. We also experimented with different setting of k in k-
nearest neighbours classifier. We tried multiple values ranging
from 1 up to 20, but we got the best results with k set to 1.
\\

To summarise, we present a comparison of achieved classification
accuracies with different methods and classifiers in
Table II. We achieved better results with Na�?ve Bayes classifier
than k-nearest neighbours. This means that Na�?ve Bayes classifier
able to learn better weights than concept weights inferred
by our method, which indicates that they are not quite optimal
yet and are subject to our next research. We can observe the
worst results achieved by our method compared to TF-IDF
in classifying documents from misc:forsale category. This
category deals with miscellaneous items for sale. These items
are rather diverse and the concept of selling them gets lost.
This causes TF-IDF vector to perform substantially better in
this category, since it captures also the words related to selling.
\\

To discuss the benefits of our approach in a real-world scenario,
we showcase sample results of extracted key-concepts
for Wikipedia article about �data structure� (see Table III).
To present our contribution in the proposed concept ranking
method, we point out the difference in results when considering
the information content and when the concept ordering ignores
it. This difference turns out to be a key-factor of performance
gain in the classification above. In the first approach, we
perform PageRank of the concepts using just the hypernymy,
holonymy and collocation relations among them (inspired by
TextRank keyword extraction). The second approach considers
also the information content of the concepts, as an analogy
to notion of inverse document frequency, which supplements
the PageRank value of the first approach in the final value of
concept relevance. We can see that using the second approach
(the right part of Table III), the results are more reasonable
compared to the more noisy concepts produced by the first
approach (the left part of Table III).
\\

\section{CONCLUSION}

We present the promising results of our key-concept extraction
algorithm. It provides very efficient representation of
document content - very concise, yet still of sufficient quality,
as we demonstrate in evaluation of classification task. Such
key-concept representation has several advantages compared
to standard keyword vector. The major drawback of keyword
vector is its representation - it is just a vector of words,
where many of them can be (and often are) ambiguous. As
for the key-concept representation, every concept has very
concrete interpretation. Since it corresponds exactly to some
WordNet synset, we can easily retrieve exact meaning of it.
Every synset contains a gloss, which may contain definition
or some example sentences of it. In addition, we know exact
relations to other synsets, like hypernym, hyponym, holonym,
meronym, etc7. With all these information at hand, we can 
easily derive the exact meaning of given synset and thus, also
of every extracted key-concept.
\\

Another advantage of our key-concept representation is
its space efficiency. We can extract high-quality key-concepts,
which are satisfyingly accurate to be used in classification. We
can summarise the main focus of given article with just a few
extracted key-concepts and still retain the conceptual notion
of its focus. Such concise representation is quite convenient
for use on the Web. It also contributes to better performance,
which is crucial for today�s big data. It could help in building
�more semantic� Web due to its relatively simple and fast
acquisition. Moreover, it should be more feasible to use in
personalisation due to its unambiguous exact interpretation and
greater information content. Last, but not least, since we used
Google N-gram corpus to approximate concept probabilities, it
allows us to perform online key-concept extraction (i.e. we do
not need to have the whole corpus in advance), which supports
the idea of using it on the Web.
\\

\section*{ACKNOWLEDGMENT}

This work was partially supported by the the Scientific
Grant Agency of Slovak Republic, grants VG1/0675/11,
VG1/0971/11 and by the Slovak Research and Development
Agency under the contract No. APVV-0208-10.
\\

%\bibliographystyle{plain}
%\bibliography{}

\end{document}